\documentclass[12pt,letterpaper]{article}
\usepackage{fullpage}
\usepackage[top=2cm, bottom=4.5cm, left=2.5cm, right=2.5cm]{geometry}
\usepackage{amsmath,amsthm,amsfonts,amssymb,amscd}
\usepackage{lastpage}
\usepackage{enumerate}
\usepackage{fancyhdr}
\usepackage{mathrsfs}
\usepackage{xcolor}
\usepackage{graphicx}
\usepackage{listings}
\usepackage{hyperref}

\hypersetup{
  colorlinks=true,
  linkcolor=blue,
  linkbordercolor={0 0 1}
}
 
\renewcommand\lstlistingname{}
\renewcommand\lstlistlistingname{}     % <-- certain figures take this
\def\lstlistingautorefname{}

\lstdefinestyle{Python}{
    language        = Python,
    frame           = lines, 
    basicstyle      = \footnotesize,
    keywordstyle    = \color{blue},
    stringstyle     = \color{green},
    commentstyle    = \color{red}\ttfamily
}

\setlength{\parindent}{0.0in}
\setlength{\parskip}{0.05in}

% Edit these as appropriate
\newcommand\course{CSE 3666}
\newcommand\hwnumber{1}                  % <-- homework number
\newcommand\NetIDa{}           % <-- NetID of person #1

\pagestyle{fancyplain}
\headheight 35pt
\lhead{Kyle Morris \\ Version 0.0.1}
\chead{\textbf{\Large --OECODE--}}
\rhead{Revised:\\ \today}
\lfoot{}
\cfoot{\small\thepage}
\rfoot{}
\headsep 1.5em

\begin{document}


\section*{Introduction \hfill}

\quad OECODE is a special language interpreted by the Operation Extinction Card Game$^{\copyright}$ that determines card effects. A code snippit can be placed in any text file, however it must be saved with the \textbf{*.oeff} file extension.

\quad This guide is for those who would like to learn from the very beginning, or to refresh themselves so they can start creating their own cards in the Card Editor. This guide provides surface-level analysis, examples, and an index of all the commands, their meaning and how to use them fluently.

\quad Throughout the text, \textbf{Rules} will be highlighted in bold. The interpreter itself follows these rules exactly, so understand them to truly grasp this language and how it is to be used.

\section{Basic Syntax}

\quad Every effect in this game can be organized and forumlated with the following syntax rule:

$$<condition(s)> [; or :] \quad target(s) \quad effect(s) \quad end/continue $$

For simplicity, OECODE uses the following convention:

$$ <0> 1-2-3-4$$

Where, 0 is one or more conditions, 1 is either ';' or ':', 2 is the targets, 3 is the actual effect, and 4 signifies the end or the continuation. Thus, the following rule:

\textbf{Rule 1: All code snippit's instructions must follow the 01234 ordering.}

If this rule is not followed, the code will not pass the interpreter's syntax-check.

\quad The .oeff file has its own structure which we will describe here in detail. The main idea is \textit{encapsulation} of the text to allow correct timing and bundeling of the data.

Let's examine the file "oecode\_intro1.oeff":

\#start\#

\&1\&

\quad \$0genercond\$ \%Enter

\quad \$1timing\$ \%Trigger

\quad \$2target\$ \%Unit

\quad \$3boost\$ \%5 \%5

\quad \$3duration\$ \%Turn \%1

\quad \$4end\$

\#end\#

Note that the name of the file must be \textbf{exactly} the same as the card it will describe.

The effect, in standard English, is:

$<$Enter$>$ ; Target Unit gains 5 ATK/DEF this turn.

\begin{itemize}
\item
	The outer-most layer, with text surrounded by " \# ", is called a \textbf{field Tokens}. Most of the time, \#start\# and \#end\# will be here.
\item
	The next layer, with text surrounded by " \& ", represents the \textbf{effect number Tokens}; this starts at 1 and can go up to 8. Cards can have 1 or up to 8 seperate effects. This is how we distinguish between them. 
\item
	The main layer, with texts surrounded by " \$ ", represents the \textbf{effect Tokens}. These vary greatly, and are discussed in greater detail later. They signify different abstract areas to apply changes to.
\item
	Following each effect Token are texts that start with " \% ". These are \textbf{Flags}, and specify to what extent changes are made, and/or to what areas they occur.
\end{itemize}

\textbf{Rule 2: Tokens must be surrounded by special characters to define order and responsibilities.}

\textit{Note: Try to read and understand the file oecode\_intro2.oeff}

\section{Field Tokens}

Currently, Field Tokens exist simply for the parser to know when to start and stop. This leaves room for more flexability in the future.
\\\\
\textbf{\#start\#}: Signifies the start of the program. 

\textbf{\#end\#}: Signifies the end of the program.

\section{Effect Tokens}

Possibly the most important section in this guide, the Effect Tokens are where true customization come into play, however they still follow \textbf{Rule 1}, and it is made evident by their names. First, some conventions:

\begin{itemize}
	\item All Effect Token definitions begin with a number. These numbers \textbf{must} be in order.
	\item Effect Tokens are usually followed by Flags that help further describe 
\end{itemize}







%Here is an example of how you can insert a figure.
%\begin{figure}[!h]
%\centering
%\includegraphics[width=0.3\linewidth]{heidi.jpg}
%\caption{Heidi attacked by a string.}
%\end{figure}

%%%% This section is required for CSE 3502 (else, comment out) %%%%

%\vfill
%\hrule
%\begin{enumerate}
%\item
%Book, online resource or person (not textbook)
%\end{enumerate}


\end{document}
